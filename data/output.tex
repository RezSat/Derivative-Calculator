
    \documentclass{article}

    %  Русский язык

    \usepackage[T2A]{fontenc}			% кодировка
    \usepackage[utf8]{inputenc}			% кодировка исходного текста
    \usepackage[english,russian]{babel}	% локализация и переносы
    \usepackage{unicode-math}

    % Рисунки
    \usepackage{graphicx, float}
    \usepackage{wrapfig}


    \title{The great derivative}
    \author{Dodo}
    \date{November 2022}


    \begin{document}
    \maketitle
    
        Welcome to derivative calculator, here is the
        step by step results of the calculations.

        \textit{A programm was given an input expression}:

    
\begin{equation}
{({\frac{({X})}{({{5}\cdot{X}})}})\cdot({{X}\cdot{4}})}
\end{equation}
Weare now working with this part:\\
\begin{equation}
{({\frac{({X})}{({{5}\cdot{X}})}})\cdot({{X}\cdot{4}})}
\end{equation}
Then let's simplify it\\
\begin{equation}
{({\frac{{X}}{{0}}})\cdot({0})}
\end{equation}
Weare now working with this part:\\
\begin{equation}
{\frac{{X}}{{0}}}
\end{equation}
Then let's simplify it\\
\begin{equation}
{\frac{{X}}{{0}}}
\end{equation}
We finally calculated the derivative of this part, lets jump up:

\begin{equation}
{\frac{({{{1}\cdot{0}}-{{X}\cdot{0}}})}{({{0}\cdot{0}})}}
\end{equation}
We finally calculated the derivative of this part, lets jump up:

\begin{equation}
{{({\frac{({{{1}\cdot{0}}-{{X}\cdot{0}}})}{({{0}\cdot{0}})}})\cdot({0})}+{({\frac{{X}}{{0}}})\cdot({0})}}
\end{equation}

        The solution is pretty simple and you definetely can do it \textbf{yourself}
        \end{document}
    