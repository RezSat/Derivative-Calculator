
    \documentclass{article}

    %  Русский язык

    \usepackage[T2A]{fontenc}			% кодировка
    \usepackage[utf8]{inputenc}			% кодировка исходного текста
    \usepackage[english,russian]{babel}	% локализация и переносы
    \usepackage{unicode-math}

    % Рисунки
    \usepackage{graphicx, float}
    \usepackage{wrapfig}


    \title{The great derivative}
    \author{Dodo}
    \date{November 2022}


    \begin{document}
    \maketitle
    
        Welcome to derivative calculator, here is the
        step by step results of the calculations.

        \textit{A programm was given an input expression}:

    
\begin{equation}
{{{X}\cdot{2}}+{{X}+{({0})ln({X})}}}
\end{equation}
Let's calculate a derivative of the one part\\
\begin{equation}
{({\frac{{1}}{{X}}})\cdot({1})}
\end{equation}
Then let's simplify it\\
\begin{equation}
{({\frac{{1}}{{X}}})\cdot({1})}
\end{equation}
Let's calculate a derivative of the one part\\
\begin{equation}
{{1}+{({\frac{{1}}{{X}}})\cdot({1})}}
\end{equation}
Then let's simplify it\\
\begin{equation}
{{1}+{({\frac{{1}}{{X}}})\cdot({1})}}
\end{equation}
Let's calculate a derivative of the one part\\
\begin{equation}
{{{1}\cdot{2}}+{{X}\cdot{0}}}
\end{equation}
Then let's simplify it\\
\begin{equation}
{2}
\end{equation}
Let's calculate a derivative of the one part\\
\begin{equation}
{{2}+{{1}+{({\frac{{1}}{{X}}})\cdot({1})}}}
\end{equation}
Then let's simplify it\\
\begin{equation}
{{2}+{{1}+{({\frac{{1}}{{X}}})\cdot({1})}}}
\end{equation}
Let's calculate a derivative of the one part\\
\begin{equation}
{{2}+{{1}+{({\frac{{1}}{{X}}})\cdot({1})}}}
\end{equation}
Then let's simplify it\\
\begin{equation}
{{2}+{{1}+{({\frac{{1}}{{X}}})\cdot({1})}}}
\end{equation}

        The solution is pretty simple and you definetely can do it \textbf{yourself}
        \end{document}
    